%% start of file `template.tex'.
%% Copyright 2006-2015 Xavier Danaux (xdanaux@gmail.com).
%
% This work may be distributed and/or modified under the
% conditions of the LaTeX Project Public License version 1.3c,
% available at http://www.latex-project.org/lppl/.


\documentclass[10pt,a4paper,sans]{moderncv}        % possible options include font size ('10pt', '11pt' and '12pt'), paper size ('a4paper', 'letterpaper', 'a5paper', 'legalpaper', 'executivepaper' and 'landscape') and font family ('sans' and 'roman')

% moderncv themes
\moderncvstyle{casual}                             % style options are 'casual' (default), 'classic', 'banking', 'oldstyle' and 'fancy'
\moderncvcolor{blue}                               % color options 'black', 'blue' (default), 'burgundy', 'green', 'grey', 'orange', 'purple' and 'red'
%\renewcommand{\familydefault}{\sfdefault}         % to set the default font; use '\sfdefault' for the default sans serif font, '\rmdefault' for the default roman one, or any tex font name
%\nopagenumbers{}                                  % uncomment to suppress automatic page numbering for CVs longer than one page

% character encoding
\usepackage[utf8]{inputenc}                       % if you are not using xelatex ou lualatex, replace by the encoding you are using
%\usepackage{CJKutf8}                              % if you need to use CJK to typeset your resume in Chinese, Japanese or Korean
\usepackage{comment}
% adjust the page margins
\usepackage[scale=0.75]{geometry}
%\setlength{\hintscolumnwidth}{3cm}                % if you want to change the width of the column with the dates
%\setlength{\makecvtitlenamewidth}{10cm}           % for the 'classic' style, if you want to force the width allocated to your name and avoid line breaks. be careful though, the length is normally calculated to avoid any overlap with your personal info; use this at your own typographical risks...

% personal data
\name{Rodrigo Guadalupe}{Chávez Jiménez}
\title{Currículum Vitae}                               % optional, remove / comment the line if not wanted
%\address{street and number}{postcode city}{country}% optional, remove / comment the line if not wanted; the "postcode city" and "country" arguments can be omitted or provided empty
\phone[mobile]{+52~(55)~41~83~42~78}                   % optional, remove / comment the line if not wanted; the optional "type" of the phone can be "mobile" (default), "fixed" or "fax"
%\phone[mobile]{+52~(55)~37~82~72~92}                   % optional, remove 
%\homepage{www.johndoe.com} 
%\social[linkedin]{john.doe}                        
%\social[twitter]{jdoe}    
\social[github]{RodGpe} 
\email{rodrigo.ch1995@gmail.com}
%\email{alejandro@ferat.com.mx}                             % optional, remove / comment the line if not wanted
%\extrainfo{ferat@ciencias.unam.mx}                 % optional, remove / comment the line if not wanted
\photo[64pt][0.4pt]{rodrigo}                       % optional, remove / comment the line if not wanted; '64pt' is the height the picture must be resized to, 0.4pt is the thickness of the frame around it (put it to 0pt for no frame) and 'picture' is the name of the picture file
%\quote{Some quote}                                 % optional, remove / comment the line if not wanted

% bibliography adjustements (only useful if you make citations in your resume, or print a list of publications using BibTeX)
%   to show numerical labels in the bibliography (default is to show no labels)
\makeatletter\renewcommand*{\bibliographyitemlabel}{\@biblabel{\arabic{enumiv}}}\makeatother
%   to redefine the bibliography heading string ("Publications")
%\renewcommand{\refname}{Articles}

% bibliography with mutiple entries
%\usepackage{multibib}
%\newcites{book,misc}{{Books},{Others}}
%----------------------------------------------------------------------------------
%            content
%----------------------------------------------------------------------------------
\begin{document}
%\begin{CJK*}{UTF8}{gbsn}                          % to typeset your resume in Chinese using CJK
%-----       resume       ---------------------------------------------------------
\makecvtitle

\section{Datos Personales}
\begin{flushleft}
\scalebox{0.9}{
\begin{tabular}{rcl}
Nombre Completo & $\;\;\;\;\;\;$ & Rodrigo Guadalupe Chávez Jiménez  \\
Fecha de Nacimiento & $\;\;\;\;\;\;$ & 12 de Diciembre de 1995 \\
Lugar de Nacimiento & $\;\;\;\;\;\;$ & Delegación Alvaro Obregón, DF. \\
Nacionalidad & $\;\;\;\;\;\;$ & Mexicana \\
Dirección & $\;\;\;\;\;\;$ & 
La Cruz \# 240, Col. Xicalhuacan  \\
&$\;\;\;\;\;\;$& Alcaldía Xochmilco, CDMX
 \\
CP & $\;\;\;\;\;\;$ & 16514 \\
CURP & $\;\;\;\;\;\;$ & CAJR951212HDFHMD08 \\
RFC & $\;\;\;\;\;\;$ & CAJR951212R85 \\
Última Actualización & $\;\;\;\;\;\;$ & 15 de marzo 2020 \\
\end{tabular}}
\end{flushleft}


\section{Formación Académica}
\cvitem{2011-2015 }{\textbf{Bachillerato}}{Escuela Nacional Preparatoria No. 1 Gabino Barreda}
\cvitem{}{}
\cvitem{2014-2015 }{\textbf{Carrera Técnica en Computación}}{Escuela Nacional Preparatoria No. 1 Gabino Barreda}
\cvitem{}{}
\cvitem{2015-2018 }{\textbf{Ingeniería en Computación}}{Facultad de Estudios Superiores Aragón, UNAM} 
\cvitem{}{}
\cvitem{2018-2020 }{\textbf{Maestría en Ciencias de la Computación} (En proceso)}{IIMAS, UNAM, Ciudad de México.} 



\section{Capacitación Laboral y Cursos}
\cventry{2017}{Seminario: Matemáticas para la Criptografía (60 hrs.)}{Laboratorio de Seguridad Informática}{{CTA Aragón}}{FES Aragón}{}
\cventry{2019}{Escuela de invierno: Ciencia de Datos}{IIMAS Mérida}{{Mérida}}{}{}
\cventry{2019}{XXXIV Coloquio Victor Neumann Lara}{COZCyT}{{Zacatecas}}{}{}
\cventry{2019}{Escuela de Verano Geometría Combinatoria y Computacional}{UAZ}{{Zacatecas}}{}{}
\cventry{2019}{18th Workshop: Routing in Mérida}{UAY}{{Mérida}}{}{}
\cventry{2019}{Congreso Nacional de la Sociedad Matemática Mexicana}{UANL}{{Monterry}}{}{}
\cventry{2020}{VI Escuela Mexicana de Invierno de Matemáticas Discretas}{CIMAT}{{Guanajuato}}{}{}
\cventry{2020}{XXXV Coloquio Victor Neumann Lara}{CECEQ}{{Querétaro}}{}{}

\section{Ponencias}

\cventry{2020}{Trayectorias ortogonales monocromáticas ajenas}{XXXV Coloquio Victor Neumann Lara}{CECEQ}{{Querétaro}}{}

%\section{Master thesis}
%\cvitem{title}{\emph{Title}}
%\cvitem{supervisors}{Supervisors}
%\cvitem{description}{Short thesis abstract}

\section{Estancias Académicas}

\cventry{2016 - 2018}{Laboratorio de Seguridad Informática}{{CTA Aragón}}{FES Aragón}{}{
	\begin{itemize}
		%\item Miembro del Proyecto Cyber Threat Intelligence
		\item Miembro del Proyecto "Sistema de Control de Acceso de Doble Factor".
		\item Miembro del Área de Desarrollo de Software Seguro.
	\end{itemize}}


\section{Experiencia}
\subsection{Apoyo a la divulgación científica}
\cventry{2019}{Miembro del comite organizador de la Semana de la Seguridad Informática $11^{a}$, $12^{a}$ y $13^{a}$ edición}{FES Aragón}{{UNAM}}{}{}
%\cvitem{2017-2018}{Miembro del comite organizador de la Semana de la Seguridad Informática $11^{a}$, $12^{a}$ y $13^{a}$ edición}{FES Aragón, UNAM.}
%\subsection{Laboral}

%\newpage
\section{Idiomas}
\cvitemwithcomment{Español}{100\%}{Idioma Nativo}
%\cvitemwithcomment{Ingles}{90\%}{TOEFL IBT, puntaje 508/560}
\cvitemwithcomment{Ingles}{80\%}{}


\section{Conocimientos y Habilidades}
\cvitem{Lenguajes de Programación}{Java, C, PHP, Javascript, Python, Ensamblador para microcontroladores}
\cvitem{}{}
\cvitem{Tecnologías web}{HTML, CSS, Angular, Electron}
\cvitem{}{}
\cvitem{Micro electrónica}{Microcontroladores, CPLD, FPGA, Arduino}
\cvitem{}{}
\cvitem{Sistemas Operativos}{Linux, Windows }
\cvitem{}{}	
\cvitem{Lenguajes Descriptores de Hardware}{VHDL, Verilog}
\cvitem{Otros}{MySQL, Docker, Struts2, Spring, \LaTeX}

\section{Intereses}
\cvitem{Seguridad Informática}{Criptografía}
\cvitem{}{}
\cvitem{Algoritmos}{Geometría computacional, Combinatoria}
\cvitem{}{}
\clearpage
\end{document}


